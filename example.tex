\documentclass{article}
\usepackage[colorlinks,citecolor=blue]{hyperref}
\usepackage{biblatex}
\addbibresource{all-biblatex.bib}
\begin{document}
% \maketitle

Here is an example using \texttt{all-biblatex.bib} with Bib\LaTeX{}.

Some example citations: classics like \textcite{chomsky.n:1957, kolmogorov.a:1968,
tesniere.l:1959, estes.w:1959}, a document with DOI and URL: \textcite{earley.j:1970},
and an arXiv document like \textcite{tay.y:2022}.

Note: \texttt{all-biblatex.bib} complies with
the format recommendations for Bib\LaTeX, whereas \texttt{all.bib} uses those for Bib\TeX. 
They both are generated from the same underlying database.
I would like to just use the Bib\LaTeX version, as it is a more modern format, but depending on the use-case, the Bib\TeX version may be required, so I'll keep generating both for now. In practice, the difference mostly means \texttt{all-biblatex.bib} has \emph{date} fields and may use Unicode characters, where \texttt{all.bib} has \emph{year}, \emph{month}, and \emph{day} fields, and uses no Unicode characters, etc.

\printbibliography
\end{document}
