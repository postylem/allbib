%! TeX program = xelatex
\documentclass{article}
\usepackage{libertinus}
\usepackage[colorlinks,citecolor=blue]{hyperref}
\usepackage[natbib, sorting=none, style=apa]{biblatex}
\addbibresource{all-biblatex.bib}
\begin{document}
% \maketitle
Example use of \texttt{all-biblatex.bib} with Bib\LaTeX{}.
Publications \citet{blachman.n:1968, kolmogorov.a:1968, tesniere.l:1959book, estes.w:1959},
one from translation \citep{levenshtein.v:1966trans},
and a document with DOI and URL: \citet{earley.j:1970}.
Also, preprints or documents from arxiv like
\citet{tay.y:2022, wu.m:2022arxiv, schijndel.m:2020psyarxiv},
and an unpublished document: \citet{narayanan.s:2004}.

Note: \texttt{all-biblatex.bib} complies with
the format recommendations for Bib\LaTeX, whereas \texttt{all.bib} uses those for Bib\TeX. 
They both are generated from the same underlying database.
I would like to just use the Bib\LaTeX version, as it is a more modern format, but depending on the use-case, the Bib\TeX version may be required, so I'll keep generating both for now. In practice, the difference mostly means \texttt{all-biblatex.bib} has \emph{date} fields and may use Unicode characters, where \texttt{all.bib} has \emph{year}, \emph{month}, and \emph{day} fields, and uses no Unicode characters, etc.

\printbibliography
\end{document}
